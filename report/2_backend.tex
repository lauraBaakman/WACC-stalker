%!TEX root = report.tex

This chapter discusses the back-end. This includes the REST API plus the database storing all the statistics data. We will first discuss the technologies that where used to build these component then how together these technologies and components form the back-end.

\section{Technology Stack}
\label{sec:2:technologyStack}
In this section we will describes as in section \ref{sec:1:technologyStack} what technologies where used and why we decided to used these above others.

\subsection{Flask}
\label{ssec:2:flask}
Flask \cite{flask} is a python micro framework for the web. Flask is based on Werkzeug \cite{werkzeug} and Jinja2 \cite{jinja2}. We have used the Flask framework as a base, but mostly used it's extensions described below. We choose Flask for our back-end because one of us already had some experience with Flask and both of us had worked (a little bit) with python before.

\subsubsection{Extensions}
To make our development easier we decided to use already existing solutions for building a REST API using Flask and MongoDb (\autoref{ssec:2:mongodb}). In this section we will describe the used extensions.

\begin{description}

\item[Flask-Restful] s an extension for Flask that adds support for quickly build- ing REST APIs. It is a lightweight abstraction that can work together with existing ORM/libraries. Because Flask-RESTful \cite{flask-restful} encourages best practices with minimal setup we were able to build a qualitatively good REST with reasonable ease.

\item[Flask-MongoKit] is and extension that builds on the Python Pymongo extension \cite{pymongo} and Flask. MongoKit adds an easy way to define MongoDb database models/documents \cite{flask-mongokit}.
\end{description}

\subsection{MongoDB}
\label{ssec:2:mongodb}
MongoDB (from``humongous'') is an open-source, document based, NoSQL database with a lot of cool features, we list a few of these that were important for the choice for MongoDB over other NoSQL database solutions \cite{mongo-db}.

\begin{description}
\item[Replication \& High Availability] A replica set in MongoDB is a group of mongod processes that maintain the same data set. Replica sets provide redundancy and high availability, and are the basis for all production deployments. Because of this feature and the requirement for the project to be fault tolerant we found this to be a positive point for choosing MongoDB. The actual implementation of this feature in our project will be discussed in \autoref{ssec:2:replication}.

\item[Map/Reduce] MongoDB provides an easy way to use map/reduce functionality. The map and reduce function can be written in JavaScript. In the Flask back-end it is very easy to apply these function to the database. This functionality was a requirement for the project and will be discussed in \autoref{ssec:2:mapreduce}.
\end{description}

\subsection{HAProxy}
\label{ssec:2:haproxy}
From the website of HAProxy \cite{ha-proxy}. HAProxy is a free, very fast and reliable solution offering high availability, load balancing, and proxying for TCP and HTTP-based applications. It is particularly suited for very high traffic web sites and powers quite a number of the world's most visited ones. Over the years it has become the de-facto standard opensource load balancer, is now shipped with most mainstream Linux distributions, and is often deployed by default in cloud platforms.

HAProxy is a load balancing solution pointed out by one of the student assistants and we chose to use this because it is according to their website the default solution for any professional cloud platform and beside fairly easy to set up. We will discuss HAProxy further in \autoref{ssec:2:loadbalancing}.

\section{Design}

This section describes the design of the back-end of the STALKER project. Because API calls to the social media API's are done in the frontend (\autoref{sec:1:design})) the back-end only has to worry about the statistics of the people who use the application to search, we call these persons, humorously, Stalkers. The people whom are being searched are stored as the Victim of a Stalker. This link is stored in what we call a Search. 

\begin{figure}
	\missingfigure{Overview image of how the back-end is structured}	
	\caption{A schematic overview of the control flow when a user logs in.}
	\label{fig:2:overview_backend}
\end{figure}

\subsection{REST API}
\todo[inline]{Possible api calls table?}
\todo[inline]{Motivation why we used a REST APi}

\subsection{Replication}
\label{ssec:2:replication}

\subsection{Map Reduce}
\label{ssec:2:mapreduce}
\todo[inline]{hmmm...}

\subsection{Load balancing}
\label{ssec:2:loadbalancing}

\section{Scalability}
REST API
Mongo replicasets enzo...

\begin{figure}
	\missingfigure{Overview image of how the back-end is structured}	
	\caption{A schematic overview of the control flow when a user logs in.}
	\label{fig:2:replica_set}
\end{figure}
